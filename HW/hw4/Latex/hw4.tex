
\documentclass[12pt]{article}

\usepackage{fancyhdr} % Required for custom headers
\usepackage{lastpage} % Required to determine the last page for the footer
\usepackage{extramarks} % Required for headers and footers
\usepackage{graphicx} % Required to insert images
\usepackage{amsmath}
\usepackage{float}
\usepackage{bm}
\usepackage{listings}
%\usepackage{lipsum} % Used for inserting dummy 'Lorem ipsum' text into the template
% Margins
\topmargin=-0.45in
\evensidemargin=0in
\oddsidemargin=0in
\textwidth=6.5in
\textheight=9.0in
\headsep=0.25in 
\linespread{1.1} % Line spacing
% Set up the header and footer
\pagestyle{fancy}
\lhead{\hmwkAuthorName} % Top left header
\chead{\hmwkClass\ : \hmwkTitle} % Top center header
\rhead{\firstxmark} % Top right header
\lfoot{\lastxmark} % Bottom left footer
\cfoot{} % Bottom center footer
\rfoot{Page\ \thepage\ of\ \pageref{LastPage}} % Bottom right footer
\renewcommand\headrulewidth{0.4pt} % Size of the header rule
\renewcommand\footrulewidth{0.4pt} % Size of the footer rule
\setlength\parindent{0pt} % Removes all indentation from paragraphs

%----------------------------------------------------------------------------------------
%	DOCUMENT STRUCTURE COMMANDS
%----------------------------------------------------------------------------------------

% Header and footer for when a page split occurs within a problem environment
\newcommand{\enterProblemHeader}[1]{
\nobreak\extramarks{#1}{#1 continued on next page\ldots}\nobreak
\nobreak\extramarks{#1 (continued)}{#1 continued on next page\ldots}\nobreak
}

% Header and footer for when a page split occurs between problem environments
\newcommand{\exitProblemHeader}[1]{
\nobreak\extramarks{#1 (continued)}{#1 continued on next page\ldots}\nobreak
\nobreak\extramarks{#1}{}\nobreak
}

\setcounter{secnumdepth}{0} % Removes default section numbers
\newcounter{homeworkProblemCounter} % Creates a counter to keep track of the number of problems

\newcommand{\homeworkProblemName}{}
\newenvironment{homeworkProblem}[1][Problem \arabic{homeworkProblemCounter}]{ % Makes a new environment called homeworkProblem which takes 1 argument (custom name) but the default is "Problem #"
\stepcounter{homeworkProblemCounter} % Increase counter for number of problems
\renewcommand{\homeworkProblemName}{#1} % Assign \homeworkProblemName the name of the problem
\section{\homeworkProblemName} % Make a section in the document with the custom problem count
\enterProblemHeader{\homeworkProblemName} % Header and footer within the environment
}{
\exitProblemHeader{\homeworkProblemName} % Header and footer after the environment
}

\newcommand{\problemAnswer}[1]{ % Defines the problem answer command with the content as the only argument
\noindent\framebox[\columnwidth][c]{\begin{minipage}{0.98\columnwidth}#1\end{minipage}} % Makes the box around the problem answer and puts the content inside
}

\newcommand{\homeworkSectionName}{}
\newenvironment{homeworkSection}[1]{ % New environment for sections within homework problems, takes 1 argument - the name of the section
\renewcommand{\homeworkSectionName}{#1} % Assign \homeworkSectionName to the name of the section from the environment argument
\subsection{\homeworkSectionName} % Make a subsection with the custom name of the subsection
\enterProblemHeader{\homeworkProblemName\ [\homeworkSectionName]} % Header and footer within the environment
}{
\enterProblemHeader{\homeworkProblemName} % Header and footer after the environment
}
   
%----------------------------------------------------------------------------------------
%	NAME AND CLASS SECTION
%----------------------------------------------------------------------------------------

\newcommand{\hmwkTitle}{Homework 4} % Assignment title
\newcommand{\hmwkDueDate}{\date} % Due date
\newcommand{\hmwkClass}{STAT\ 580} % Course/class
\newcommand{\hmwkAuthorName}{Haozhe Zhang} % Your name

%----------------------------------------------------------------------------------------

\begin{document}


	\begin{homeworkProblem}
		\noindent \textbf{C code:}
		\lstinputlisting[language=C]{../hw4_1.c}
	\end{homeworkProblem}


\begin{homeworkProblem}
	(a)
	\begin{eqnarray*}
    \int_{0}^{\infty}(x^{2}+5)xe^{-x}dx \approx 10.99763 \pm 0.2028816
	\end{eqnarray*}
	
	(b)
	\begin{eqnarray*}
		\int_{0}^{1}\int_{-\infty}^{\infty}e^{-x^{2}}\cos(xy)dxdy \approx 1.634335
	\end{eqnarray*}
	
	(c)
	\begin{eqnarray*}
		\int_{0}^{\infty}\frac{3}{4}x^{4}e^{-x^{3}/4}dx \approx 2.271137 \pm 0.00957
	\end{eqnarray*}
	\noindent \textbf{R code:}
	\lstinputlisting[language=R]{../hw4_2.R}
\end{homeworkProblem}


\begin{homeworkProblem}
	\begin{itemize}
		\item $\nu=0.1$: $I \approx 0.185114 \pm 0.07967168$;
		\item $\nu=1$: $I \approx 0.1363114 \pm 0.00121565$;
		\item $\nu=10$: $I \approx 0.1386195 \pm 0.004512526$.
	\end{itemize}

	\noindent \textbf{R code:}
	\lstinputlisting[language=R]{../hw4_3.R}
\end{homeworkProblem}

\begin{homeworkProblem}
	(a)
$\hat{I} =  0.6977335$
	
	(b)
$E\{c(U)\} = 1.5$. The optimal value of $b = -0.4779221$. $\hat{I}_{CV} = 0.6927941$.
	
	(c)
	$$\widehat{var}(\hat{I}_{MC}) = 1.295893\times 10^{-5} > \widehat{var}(\hat{I}_{CV}) = 4.006056\times 10^{-7} $$
	(d) We define a new function $c_{1}(x)=\sqrt{1+x}$. The new estimator is denoted by
	\begin{eqnarray*}
	\hat{I}_{new} = \frac{1}{n}\sum_{i=1}^{n}h(U_{i}) - b\left[\frac{1}{n}\sum_{i=1}^{n}c_{1}(U_{i})-E\{c_{1}(U)\}\right].
	\end{eqnarray*}
The variance of $\hat{I}_{new}$ is smaller than the variance of $ \hat{I}_{CV}$ because

\begin{eqnarray*}
\left|\mathrm{Corr}\left(\frac{1}{1+x}, \sqrt{1+x}\right)\right| &=& \frac{E\frac{1}{\sqrt{1+x}}-E\frac{1}{1+x}E\sqrt{1+x}}{\sqrt{var(\frac{1}{1+x})var(\sqrt{1+x})}}=0.990998\\
\left|\mathrm{Corr}\left(\frac{1}{1+x}, 1+x\right)\right| &=& \frac{E1-E\frac{1}{1+x}E(1+x)}{\sqrt{var(\frac{1}{1+x})var(1+x)}}=0.9841661 < 0.990998
\end{eqnarray*}
	
	
	\noindent \textbf{R code:}
	\lstinputlisting[language=R]{../hw4_4.R}
\end{homeworkProblem}
\end{document}
