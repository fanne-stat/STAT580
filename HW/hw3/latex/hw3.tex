
\documentclass[12pt]{article}

\usepackage{fancyhdr} % Required for custom headers
\usepackage{lastpage} % Required to determine the last page for the footer
\usepackage{extramarks} % Required for headers and footers
\usepackage{graphicx} % Required to insert images
\usepackage{amsmath}
\usepackage{float}
\usepackage{bm}
\usepackage{listings}
%\usepackage{lipsum} % Used for inserting dummy 'Lorem ipsum' text into the template
% Margins
\topmargin=-0.45in
\evensidemargin=0in
\oddsidemargin=0in
\textwidth=6.5in
\textheight=9.0in
\headsep=0.25in 
\linespread{1.1} % Line spacing
% Set up the header and footer
\pagestyle{fancy}
\lhead{\hmwkAuthorName} % Top left header
\chead{\hmwkClass\ : \hmwkTitle} % Top center header
\rhead{\firstxmark} % Top right header
\lfoot{\lastxmark} % Bottom left footer
\cfoot{} % Bottom center footer
\rfoot{Page\ \thepage\ of\ \pageref{LastPage}} % Bottom right footer
\renewcommand\headrulewidth{0.4pt} % Size of the header rule
\renewcommand\footrulewidth{0.4pt} % Size of the footer rule
\setlength\parindent{0pt} % Removes all indentation from paragraphs

%----------------------------------------------------------------------------------------
%	DOCUMENT STRUCTURE COMMANDS
%----------------------------------------------------------------------------------------

% Header and footer for when a page split occurs within a problem environment
\newcommand{\enterProblemHeader}[1]{
\nobreak\extramarks{#1}{#1 continued on next page\ldots}\nobreak
\nobreak\extramarks{#1 (continued)}{#1 continued on next page\ldots}\nobreak
}

% Header and footer for when a page split occurs between problem environments
\newcommand{\exitProblemHeader}[1]{
\nobreak\extramarks{#1 (continued)}{#1 continued on next page\ldots}\nobreak
\nobreak\extramarks{#1}{}\nobreak
}

\setcounter{secnumdepth}{0} % Removes default section numbers
\newcounter{homeworkProblemCounter} % Creates a counter to keep track of the number of problems

\newcommand{\homeworkProblemName}{}
\newenvironment{homeworkProblem}[1][Problem \arabic{homeworkProblemCounter}]{ % Makes a new environment called homeworkProblem which takes 1 argument (custom name) but the default is "Problem #"
\stepcounter{homeworkProblemCounter} % Increase counter for number of problems
\renewcommand{\homeworkProblemName}{#1} % Assign \homeworkProblemName the name of the problem
\section{\homeworkProblemName} % Make a section in the document with the custom problem count
\enterProblemHeader{\homeworkProblemName} % Header and footer within the environment
}{
\exitProblemHeader{\homeworkProblemName} % Header and footer after the environment
}

\newcommand{\problemAnswer}[1]{ % Defines the problem answer command with the content as the only argument
\noindent\framebox[\columnwidth][c]{\begin{minipage}{0.98\columnwidth}#1\end{minipage}} % Makes the box around the problem answer and puts the content inside
}

\newcommand{\homeworkSectionName}{}
\newenvironment{homeworkSection}[1]{ % New environment for sections within homework problems, takes 1 argument - the name of the section
\renewcommand{\homeworkSectionName}{#1} % Assign \homeworkSectionName to the name of the section from the environment argument
\subsection{\homeworkSectionName} % Make a subsection with the custom name of the subsection
\enterProblemHeader{\homeworkProblemName\ [\homeworkSectionName]} % Header and footer within the environment
}{
\enterProblemHeader{\homeworkProblemName} % Header and footer after the environment
}
   
%----------------------------------------------------------------------------------------
%	NAME AND CLASS SECTION
%----------------------------------------------------------------------------------------

\newcommand{\hmwkTitle}{Homework 3} % Assignment title
\newcommand{\hmwkDueDate}{\date} % Due date
\newcommand{\hmwkClass}{STAT\ 580} % Course/class
\newcommand{\hmwkAuthorName}{Haozhe Zhang} % Your name

%----------------------------------------------------------------------------------------

\begin{document}

\begin{homeworkProblem}
	(a)
	\begin{eqnarray*}
	P(U\leq r(x))&=& E(I(U\leq r(x)))= E(E(I(U\leq r(x))|x)) = E(r(x))\\
	&=& \int \frac{q(x)}{\alpha g(x)}g(x)dx = \frac{1}{\alpha}\int q(x)dx
	\end{eqnarray*}
	
	(b)
	\begin{eqnarray*}
   P(x \in A, U\leq r(x)) &=& E(I(U\leq r(x))I(x\in A))=E(r(x)I(x\in A))\\
   &=& \int_{A} \frac{q(x)}{\alpha g(x)}g(x)dx = \frac{1}{\alpha}\int_{A} q(x)dx
	\end{eqnarray*}
	
	\begin{eqnarray*}
 P(Y\in A) = P(x \in A |U\leq r(x)) = \frac{P(x \in A, U\leq r(x)) }{P(U\leq r(x))}= \int_{A}\frac{q(x)}{\int q(x)dx}dx
	\end{eqnarray*}
\end{homeworkProblem}

\begin{homeworkProblem}
	(a)
	\begin{equation}
		\int (2x^{\theta-1}e^{-x} + x^{\theta -1/2}e^{-x})dx = 2\Gamma(\theta) + \Gamma(\theta+\frac{1}{2})
	\end{equation}
	(b)
	\begin{eqnarray*}
	 g(x) &=& \omega_{1}g_{1}(x) + \omega_{2}g_{2}(x)\\
	 &=& \frac{2\Gamma(\theta)}{2\Gamma(\theta) + \Gamma(\theta+\frac{1}{2})} \left(\frac{1}{\Gamma(\theta)}x^{\theta-1}e^{-x}\right) + \frac{\Gamma(\theta+\frac{1}{2})}{2\Gamma(\theta) + \Gamma(\theta+\frac{1}{2})}
	 \left(\frac{1}{\Gamma(\theta+1/2)}x^{\theta-1/2}e^{-x}\right) 
	\end{eqnarray*}
	(c)
	Algorithm:
	\begin{enumerate}
		\item Sample $U \sim Ber\left(\frac{2\Gamma(\theta)}{2\Gamma(\theta) + \Gamma(\theta+\frac{1}{2})}\right)$;
		\item If $U=1$, sample $X \sim Gamma(1,\theta)$; if $U=0$, sample $X \sim Gamma(1,\theta+1/2)$. Then, $X$ is a sample from $g(x)$.
	\end{enumerate}
	(d) It can be shown that $f(x)\leq g(x)$ for $x>0$.\\
	Algorithm:
	\begin{enumerate}
		\item Sample $X \sim g(x)$ and $U \sim Unif(0,1)$ independently;
		\item If $U > \frac{f(x)}{g(x)}$, then go to Step 1; otherwise, return $X$. The returned value is a random variable from $f(x)$.
	\end{enumerate}
	
\end{homeworkProblem}

\begin{homeworkProblem}
	\noindent \textbf{C code:}
	\lstinputlisting[language=C]{../hw3_3.c}
\end{homeworkProblem}

\begin{homeworkProblem}
	\noindent \textbf{C code:}
	\lstinputlisting[language=C]{../hw3_4.c}
\end{homeworkProblem}


\begin{homeworkProblem}
	\noindent \textbf{C code:}
	\lstinputlisting[language=C]{../hw3_5.c}
\end{homeworkProblem}

\end{document}
